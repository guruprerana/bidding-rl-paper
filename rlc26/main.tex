% RLJ main.tex Version 2025.1

\documentclass[10pt]{article} % For LaTeX2e

%%%%%%%%%%%%%%%%%%%%%%%%%%%%%%%%%%%%%%%%%%%%%%%%%%%%%%%%%%%%%%%%
% AUTHOR: Select ONE option:
%      [accepted]{rlj} --> for camera ready (after peer review, if accepted)
%      {rlj}           --> for submission
%      [preprint]{rlj} --> to de-anonymize and remove references to RLJ/RLC
%%%%%%%%%%%%%%%%%%%%%%%%%%%%%%%%%%%%%%%%%%%%%%%%%%%%%%%%%%%%%%%%
\usepackage{rlj}           % Should be uncommented for submission
%\usepackage[accepted]{rlj} % Should be uncommented for the camera-ready
%\usepackage[preprint]{rlj} % Should be uncommented for preprint versions

%%%%%%%%%%%%%%%%%%%%%%%%%%%%%%%%%%%%%%%%%%%%%%%%%%%%%%%%%%%%%%%%
% WARNING: The following packages are already included in the
%          rlj.sty style file:
%
%  1. fancyhdr  - For controlling header/footers
%  2. natbib    - For formatting the bibliography
%  3. enumitem  - To customize the appearance of lists
%  4. fontenc (with option [T1]) - Allows for proper hyphenation and accents
%  5. times     - Times new roman font
%  6. ragged2e  - Used to justify text
%  7. tcolorbox - Used to create boxes on cover page
%  8. hyperref  - Configures hyperlinks throughout (e.g., links to references)
%  9. xcolor    - Used to define custom colors for links and boxes
%  10. amsmath  - Not used, but conflicts with lineno, so we include (and patch) it for authors
%  11. etoolbox - Included in the amsmath + lineno patch
%  12. lineno   - For adding line numbers when in submission
%
% You do not need to include them again in your main.tex.
% Including them again may lead to conflicts or compilation errors.
% Additionally, avoid loading packages that might conflict with these.
%%%%%%%%%%%%%%%%%%%%%%%%%%%%%%%%%%%%%%%%%%%%%%%%%%%%%%%%%%%%%%%%

%%%%%%%%%%%%%%%%%%%%%%%%%%%%%%%%%%%%%%%%%%%%%%%%%%%%%%%%%%%%%%%%
% Recommended (but not required) packages
%%%%%%%%%%%%%%%%%%%%%%%%%%%%%%%%%%%%%%%%%%%%%%%%%%%%%%%%%%%%%%%%
\usepackage{amssymb}            % Defines common symbols like \mathbb R
\usepackage{mathtools}          % Extends amsmath, providing common math tools
\usepackage{mathrsfs}           % Enables \mathscr, which can work in cases that \mathcal does not
%\mathtoolsset{showonlyrefs}     % Only number equations that are referenced (optional)
\usepackage{graphicx}           % For including images
\usepackage{subcaption}         % Allows for the use of subfigures and subcaptions
\usepackage[space]{grffile}     % For spaces in image names
\usepackage{url}                % For displaying URLs
\usepackage{lipsum}             % For placeholder text

%%%%%%%%%%%%%%%%%%%%%%%%%%%%%%%%%%%%%%%%%%%%%%%%%%%%%%%%%%%%%%%%
% AUTHOR: Fill in the following meta-data
%%%%%%%%%%%%%%%%%%%%%%%%%%%%%%%%%%%%%%%%%%%%%%%%%%%%%%%%%%%%%%%%

\usepackage{amsthm}

\newtheorem{definition}{Definition}

\newcommand{\dist}{\mathbb{D}}
\newcommand{\N}{\mathbb{N}}
\newcommand{\Z}{\mathbb{Z}}
\newcommand{\Q}{\mathbb{Q}}
\newcommand{\R}{\mathbb{R}}
\newcommand{\C}{\mathbb{C}}
\newcommand{\abs}[1]{\left|#1\right|}
\newcommand{\momdp}{MO-MDP}
\newcommand{\mabmdp}{MAB-MDP}


% Enter the title of your paper:
\title{Divide and Coordinate: A Multi-Policy Framework for Multi-Objective Reinforcement Learning}

% The "running title" will be displayed in the header on every-other page.
% It is typically either the same as the title or a shorter version of the title.
% Enter your running title here:
\setrunningtitle{Multi-Objective RL With Multi-Agent Bidding}

% WARNING: Authors must not appear in the submitted version. They should be hidden
% as long as the rlj package is used without the [accepted] or [preprint] options.
% Non-anonymous submissions will be rejected without review.

% Enter the author names below. 
% NOTE: Denote affiliations using superscripts as in the provided example.
% NOTE: Use \textscript{1,2,3} instead of $^{1,2,3}$.
%       - Failure to do so will cause affiliation superscripts to appear on the cover page for camera-ready and preprint versions.
\author{Guruprerana Shabadi\textsuperscript{1}, Kaushik Mallik\textsuperscript{2}}

% NOTE: For camera-ready and preprint versions, the cover page includes author names but not affiliations.
% It automatically removes the superscripts for affiliations.
% If the automatic process breaks (e.g., if an author name should include a superscript), you can manually define the author string to appear on the cover page by uncommenting the following line.
%\coverPageAuthor{Marlos C. Machado, Philip S. Thomas, Lorem Ipsum}

% Author emails, which can be clustered if they have shared endings as in this example
\emails{shabadi@seas.upenn.edu, kaushik.mallik@imdea.org}

% Author affiliations, in the order the occur
% The inclusion of state/province, etc. is optional.
% The inclusion of multiple affiliations is optional.
%   - List multiple affiliations with comma-separated numbers as in the example.
\affiliations{
$^{1}$\textbf{University of Pennsylvania, United States}\\
$^{2}$\textbf{IMDEA Software Institute, Spain}
% The following two lines are optional and can be commented out
% \par % If including additional comments like below, use \par to add some whitespace. 
% $^\dagger$ Additional comments can be added like this, e.g., indicating equal contribution
}

\contribution{
    % Contribution
    Provide a succinct but precise list of the contribution(s) of the paper. Use contextual notes to avoid implications of contributions more significant than intended and to clarify and situate the contribution relative to prior work (see the examples below). If there is no additional context, enter ``None''. Try to keep each contribution to a single sentence, although multiple sentences are allowed when necessary. If using complete sentences, include punctuation. If using a single sentence fragment, you may omit the concluding period. A single contribution can be sufficient, and there is no limit on the number of contributions. Submissions will be judged mostly on the contributions claimed on their cover pages and the evidence provided to support them. Major contributions should not be claimed in the main text if they do not appear on the cover page. Overclaiming can lead to a submission being rejected, so it is important to have well-scoped contribution statements on the cover page.
    }
    {
    % Caveat/Context:
    None
    }

\contribution{
    % Contribution
    The submission template for submissions to RLJ/RLC 2025
    }
    {
    % Caveat/Context:
    Built from previous RLC/RLJ, ICLR, and TMLR submission templates
    }

\contribution{
    % Contribution
    \textit{{[}Example of one contribution and corresponding contextual note for the paper ``Policy gradient methods for reinforcement learning with function approximation'' \citep{Sutton2000}.]}\\ This paper presents an expression for the policy gradient when using function approximation to represent the action-value function.
    }
    {
    % Context:
    Prior work established expressions for the policy gradient without function approximation \citep{Williams1992}.
    }

% Include a list of keywords for the topic of the paper:
\keywords{Multi-Objective RL, Multi-Agent RL} % Your keywords

% Define the summary that appears on the cover page.
\summary{The summary appears on the cover page. Although it can be identical to the abstract, it does not have to be. One might choose to omit the stated contributions in the Summary, given that they will be stated in the box below. The original abstract may also be extended to two paragraphs. The authors should ensure that the contents of the cover page fit entirely on a single page. The cover page does \textbf{not} count towards the 8--12 page limit.

\lipsum[1]
}

%%%%%%%%%%%%%%%%%%%%%%%%%%%%%%%%%%%%%%%%%%%%%%%%%%%%%%%%%%%%%%%%
%% Begin document, create title and abstract
%%%%%%%%%%%%%%%%%%%%%%%%%%%%%%%%%%%%%%%%%%%%%%%%%%%%%%%%%%%%%%%%
\begin{document}

\makeCover  % Create the cover page
\maketitle  % Make the title section

\begin{abstract}
We study multi-objective reinforcement learning settings in which objectives may appear or disappear at runtime. 
We propose a modular framework that incrementally updates behavior without retraining the entire system.
Each objective is supported by a selfish local policy, and coordination is achieved through a novel \emph{auction}-based mechanism: 
policies bid for the right to execute their actions, with bids reflecting the urgency of the current state. 
The highest bidder selects the action, enabling a dynamic and interpretable trade-off among objectives. 
To make this possible, each local policy must not only optimize its own objective, but also reason about the presence of other goals and learn to produce calibrated bids that reflect relative priority. 
When objectives change, the system adapts by simply adding or removing the corresponding policies.
We instantiate this approach using proximal policy optimization (PPO). 
Experiments on a robotic path-planning task with dynamic targets and the Atari game Assault demonstrate substantial performance gains and significantly reduced sample complexity.
\end{abstract}
\section{Introduction}

A majority of real-world control problems require fulfilling more than one objectives simultaneously, where the objectives could be partly contradictory to each other.
We consider the setting where objectives appear or disappear at deployment time, and no prior information about their arrival is available.
Our running example is a coffee serving robot in an office building, where new coffee requests could appear from any place at any time, old requests could disappear even before being served, and the robot must react to updated objectives as quickly as possible.
In general, any number of requests could be active at any time, and we do not assume any prior knowledge about their distribution.
Our goal is to design lightweight, adaptive policies that update their behavior in light of the evolving objectives.
Despite having been studied in the planning literature~\citep{?}, this class of problems lack support in (model-free) reinforcement learning (RL), and the existing solutions from the planning setting do not readily extend.

We propose a novel policy adaptation algorithm for this class of problems.
The heart of our approach is a compositional design of policies for multi-objective RL problems, where each objective is served using an independent local policy, and coordination is achieved via an online auction-based mechanism: 
policies bid for the right to execute their actions, with bids reflecting the urgency of the current state for fulfilling the respective objectives.
The highest bidder selects the action, enabling a dynamic and interpretable trade-off among objectives.
For example, imagine a situation that the coffee serving robot is approaching an intersection of two corridors, there are two active requests and two independent local policies for serving them, the first policy needs to go left, while the second policy needs to go right.
Clearly a trade-off is unavoidable, and which of the policies is prioritized is decided based on auctions.

Given this compositional design, every time an objective appears or disappears, only the responsible policy needs to be added or removed, eliminating the need for modifying the whole system.
Moreover, when the objectives come from the same parameterized family, like the set of all possible coffee requests parameterized by the request locations, we can design a single universal policy for this entire parameterized family, so that new additions require adding an identical copy of the policy, offer us instant adaptation.
In contrast, a monolithic policy that would serve all objectives lacks this flexibility, and an attempt to train such a policy with variable number of objectives causes a dramatic degradation of training stability as well as the performance (the loss); we demonstrate this in our experiments.

We show how classical RL policies can be extended with bidding capabilities.
While the obvious first step is to extend the action space with numeric bid values, there are three key challenges:

\textbf{Challenge I: enforcing honest bids.}
We must ensure that policies bid only in proportion to their urgency, because otherwise, there is a risk of obtaining ``dishonest'' policies that constantly try to block others by overbidding.
To circumvent this issue, during training, we require the policies to pay penalties proportional to their bid values.
\KM{I suggest, we just stick to one kind of bidding, instead of three, if in the end all-pay remains superior in all experiments.}
This way, it is against the interest of policies to bid too high, because otherwise, the net benefit of achieving their objectives would be lost.

\textbf{Challenge II: achieving environment awareness.}
The bidding tactic of policies should not only depend on the current state and the own objective, but also needs to account for other objectives for maximal effectiveness.
For instance, if two coffee requests appear at nearby locations, the respective policies need not bid too high to compete against the opponent.
Dually, if two requests arrive from opposite directions, the bid of the more urgent policy must be high enough to counteract the opponent.
Therefore, the policies must account for the objectives of opponents to bid effectively.
Our approach is to model the local policy synthesis problem as an instance of multi-agent nonzerosum  games, where each agent is responsible for an individual objective, and needs to learn a policy that fulfills its objective against the opponents.

\textbf{Challenge III: unbounded objective count.} 
Since the number of objectives seen during deployment cannot be predicted during training, the game-theoretic approach for environment-aware policies falls apart, because now we face a variable number of opponents for which no solution is known.
We use an attention network that transforms an arbitrary number of opponent objectives to a fixed encoding, which is then fed as the input to the RL policy.
Each local policy is equipped with its own attention network, which is co-trained with the actual policy for an increasing number of opponent objectives.
\KM{If we want to make the unbounded objectives a central feature, we should restrict ourselves to objectives from the same family. Otherwise, the attention pooling and the game-theoretic training approach does not work. While this is indeed a restrictive case, it is well-motivated and I do not see it as a weakness.}

We implemented our framework using the proximal policy optimization (PPO) as the base learning algorithm.
Using the Atari Assault and a gridworld-based path-planning task, we demonstrate the superior training stability and the performance of our policies in comparison with the baseline monolithic policies.
\todo{Summarize the main take-away: can we give some statistics, like ``the training time was 2x faster'' or the ``loss was 2x smaller''?}
\subsection{Related work}

A large body of work in multi-objective reinforcement learning (MORL) relies on scalarization, aggregating multiple reward functions into a single scalar objective so that standard single-objective RL algorithms can be applied. The simplest scalarization method is a weighted sum of individual rewards~\citep{gass1955computational}, though richer nonlinear scalarization functions have also been proposed~\citep{van2013scalarized}. A key limitation of scalarization is that the relative importance induced by the aggregation function may not align with the designer’s true intent. This mismatch can initiate a tedious debugging cycle, particularly in large-scale systems~\citep{hayes2022practicalguidemorl}. In contrast, our approach achieves a trade-off between reward components without collapsing them into a fixed scalar objective.

Other works pursue trade-offs by fixing a specific optimality criterion. Common choices include Pareto optimality~\citep{van2014multi} and its approximations~\citep{pirotta2015multi}, as well as fairness-based criteria across reward functions~\citep{park2024max,byeon2025multi,siddique2020learning}. These approaches typically learn a single monolithic policy that satisfies the chosen criterion. By contrast, our objective is to learn independent, selfish local policies for each reward component and compose them at runtime in a principled manner, thereby preserving modularity while still achieving a coherent global trade-off.

Relatively few works study distributed local policies for multiple rewards. A notable example is W-learning~\citep{humphrys1995w} and its deep RL extension~\citep{rosero2024multi}, where separate selfish policies are trained alongside meta-policies (W-functions) that assign each state a score reflecting its urgency. At runtime, the policy with the highest score is selected. Other approaches employ alternative aggregation mechanisms, such as ranked voting over actions~\citep{mendez2019multi}, or fixed aggregation rules like summing action values across agents~\citep{russell2003q}. While conceptually related, our approach is technically simpler: it relies on an engineered reward structure that enables the use of standard learning algorithms (e.g., PPO) without additional meta-policies or complex aggregation schemes. Furthermore, to the best of our knowledge, we are the first to introduce the incremental MORL setting, in which reward components can be added or removed at runtime.

The idea of bidding-based selfish policies originates from analogous techniques for multi-objective path planning problems on finite graphs~\citep{avni2024auction}, as well as from the broader literature on bidding games~\citep{lazarus1999combinatorial,avni2019infinite,avni2025bidding}. These works study strategic interaction in finite arenas, where adversarial players bid for the right to determine the next move from a shared action space in pursuit of their objectives. Although these works provide strong theoretical guarantees, they do not naturally extend to infinite arenas. Moreover, players in such games are typically budget-constrained, and the central question concerns the minimum budget required to win. In contrast, we consider infinite arenas and eliminate explicit budget constraints by incorporating bidding rewards and penalties directly into the learning framework.

%%% ------- List of all related works collected by Guru --------------------
%\begin{enumerate}
%    \item Fairly comprehensive reference survey of multi-objective RL:~\citet{hayes2022practicalguidemorl}
%    \item Reference for definitions of multi agent RL:~\citet{bucsoniu2010multiagentoverview}.
%    \item W-learning and more recent Deep W learning:~\citet{humphrys1995w,rosero2024multi}
%    \item General multi-objective deep RL works.~\citet{nguyen2020multi} introduce a multi-policy DQN algorithm to learn multiple policies in parallel such that they have access to any possible linear weighting of the objectives. Also has a good lit review from where I got a few of these other citations.
%    \item Multi objective Q learning (tabular):~\citet{van2013scalarized},~\citet{van2014multi} for pareto Q learning (maintains a set of policies and updates them), local search after normal learning:~\citet{van2014novel}
%    \item Use gradient approximation to iteratively build parametrized policies representing pareto frontier~\citet{pirotta2015multi}
%    \item Maximin (fairness) MORL:~\citet{park2024max,byeon2025multi}, fair deep RL:~\citet{siddique2020learning}.
%    \item These two are only vaguely related and look at decomposing an objective into multiple smaller objectives: Q-function decomposition for a single agent with multiple reward components:~\citet{russell2003q} and on a similar note hybrid reward architecture:~\citet{van2017hybrid}
%\end{enumerate}
%%% --------------------------------------------------------------------------

\section{Preliminaries: Multi-Objective MDPs}
\label{sec:preliminaries}

Mainstream RL algorithms consider Markov decision processes (MDP) equipped with a \textit{single} reward function, pertaining to a single task or \textit{objective} for the system.
In reality, a majority of real-world applications of RL requires satisfying multiple, partly contradictory objectives.
We model such multi-objective decision-making problems using multi-objective MDPs (MO-MDP), as formally defined below.
Intuitively, an MO-MDP has the exact same syntax as a regular MDP, except that it now has multiple reward functions pertaining to the different objectives.
We formalize \momdp below.
We will use the notation $\dist(\Sigma)$ to represent the set of all probability distributions over a given alphabet $\Sigma$.

\begin{definition}[\momdp]\label{def:MO-MDP}
    A multi-objective Markov decision process (\momdp) with \(m \in \Z_{>0}\) objectives is specified by a tuple \(\mathcal{M} = (S, A, T, \mathbf{R}, \mu_0)\), where
    \begin{itemize}
        \item \(S\) is the set of states,
        \item \(A\) is the set of actions,
        \item \(T : S \times A \to \dist(S)\) is the transition function mapping a state-action pair to a distribution over the successor states,
        \item \(\mathbf{R}=\{R^i : S \times A \times S \to \R_{\geq 0}\}_{i\in [1;m]}\) is the set of reward functions, and
        \item \(\mu_0 \in \dist(S)\) is the initial state distribution.
    \end{itemize}
\end{definition}

The notions of policies and paths induced by them are exactly the same as in classical MDPs, which we briefly recall below.
First, we introduce some notation.
Given an alphabet $\Sigma$, we will write $\Sigma^*$ and $\Sigma^\omega$ to denote the set of every finite and infinite word over $\Sigma$, respectively, and will write $\Sigma^\infty = \Sigma^*\cup \Sigma^\omega$.
Given a word $w = \sigma_0\sigma_1\ldots\in \Sigma^\infty$, and given a $t\geq 0$ that is not larger than the length of $w$, we will write $w_t$ and $w_{0:t}$ to denote respectively the $t$-th element of $w$, i.e., $w_t=\sigma_t$, and the prefix of $w$ up to the $t$-th element, i.e., $w_{0:t} = \sigma_0\ldots\sigma_t$.


A \textit{policy} in an \momdp~\(\mathcal{M}\) is a function \(\pi : (S \times A)^* \times S \to \dist(A)\) that maps a history of state-action pairs and the current state to a distribution over actions. 
A \emph{path} on $\M$ induced by $\pi$ is a sequence $\rho=(s_0,a_0)(s_1,a_1),\ldots\in (S\times A)^\infty$ such that for every $t\geq 0$, (1)~the probability that the action $a_{t+1}$ is picked by $\pi$ based on the history is positive, i.e., $\pi(\rho_{0:t},s_{t+1})(a_{t+1}) > 0$, and (2)~the probability of moving to the state $s_{t+1}$ from $s_t$ due to action $a_t$ is positive, i.e., $T(s_t,a_t)(s_{t+1})>0$.
A path can be either finite or infinite, and we will write $\paths(\M,\pi)$ to denote the set of all infinite paths fo $\M$ induced by $\pi$.
Given a finite path $\rho=(s_0,a_0)\ldots (s_t,a_t)$, the probability that $\rho$ occurs is given by: $\mu_0(s_0)\cdot\prod_{k=0}^{t-1} T(s_k,a_k)(s_{k+1})\cdot \pi(\rho_{0:k},s_{k+1})(a_{k+1})$.
This can be extended to a probability measure over the set of all infinite paths in $\M$ using standard constructions, which can be found in the literature~\citep{baier2008principles}.
Given a measurable set of paths $\Omega$ and a function $f\colon \paths(\M,\pi)\to \mathbb{R}$, we will write $\P^{\M,\pi}[\Omega]$ and $\E^{\M,\pi}[f]$ to denote, respectively, the probability measure of $\Omega$ and the expected value of $f$ evaluated over random infinite paths.

We will use the standard discounted reward objectives, where we fix $\gamma \in (0,1)$ as a given discounting factor.
Let $\rho = (s_0,a_0)(s_1,a_1),\ldots \in \paths(\M,\pi)$ be an infinite path induced by $\pi$.
Define the discounted sum function, mapping $\rho$ to the discounted sum of the associated rewards: $f_{\mathrm{ds}}^i(\rho)\coloneqq\sum_{t=0}^\infty \gamma^t\cdot R^i(s_t,a_t)$.
The \emph{$i$-value} of the policy $\rho$ for $\M$ is the expected value of the discounted sum of the $i$-th reward we can secure by executing $\rho$ on $\M$, written as $\val^{\M,i}(\pi) = \E^{\M,\pi}[f_{\mathrm{ds}}^i]$.
The \emph{optimal} policy for $R^i$ for a given $i\in [1;m]$ is the policy that maximizes the $i$-value.
When the reward index $i$ is unimportant, we will refer to every element of the set $\{\val^{\M,i}\}_{i\in [1;m]}$ as a \emph{value component}.

When the \momdp $\M$ is clear from the context, we will drop it from all notation and will simply write $\paths(\pi)$, $\P^{\pi}$, $\E^{\pi}$, and $\val^i$.

It is known that \emph{memoryless} (aka, stationary) policies suffice for maximizing single discounted reward objectives, where a policy $\pi$ is called memoryless if the proposed action only depend on the current state.
In other words, given every pair of finite paths $\rho,\rho'$ both ending at the same state, the probability distributions $\pi(\rho)$ and $\pi(\rho')$ are identical.

Unlike classical single-objective MDPs, the optimal policy synthesis problem for \momdp requires fixing one of many possible optimality criteria.
Many possibilities exist, including pareto optimality, requiring a solution where none of the value components could be unanimously improved without hurting the others; weighted social welfare, requiring a weighted sum of the value components be maximized; and fairness, requiring the minimum attained value by any value component is maximized.
\todo{Give some citations for each category.}

\section{Auction-Based Compositional RL on Multi-Objective MDPs}

We consider the compositional approach to policy synthesis for \momdps, where we will design a selfish, \emph{local} policy maximizing each individual value component, towards the fulfillment of some required global coordination requirements.
The main crux is in the composition process, where each local policy may propose a different action, but the composition must decide one of the actions that will be actually executed.
Importantly, the composition must be implementable in a distributed manner, meaning we will \emph{not} use any global policy that would pick an action by analyzing all local policies and their reward functions.
\todo{running example}

\subsection{The Framework}
We present a novel \textit{auction}-based RL framework for compositional policy synthesis for \momdps.
In our framework, not only do the local policies emit actions, but also they \emph{bid} for the privilege of executing their actions for a given number of time steps $\tau\in \mathbb{N}_{>0}$ in future.
The bids are all nonnegative real numbers, and the highest bidder's actions get executed for the following $\tau$ consecutive steps, with bidding ties being resolved uniformly at random.
The policy whose actions are executed is referred to as the \emph{winning} policy, and it must pay a bidding \emph{penalty} that equals to its bid amount; this is to discourage overbidding.
The policies whose actions are not executed are called the \emph{losing} policies, and we consider three different settings for the ``payment'' they must make:
\begin{description}
	\item[\richman:] the winning policy pays the bidding penalty and the losing policies earn bidding rewards equal to their respective bid values;
	\item[\poorman:] the winning policy pays the bidding penalty and the losing policies are unaffected (i.e., neither earn bidding rewards nor pay bidding penalties);
	\item[\allpay:] all policies pay bidding penalties equal to their respective bid values.
\end{description}
While penalizing the winner discourages overbidding, the situation with the losers is more subtle.
In the \richman setting, by rewarding the losers, we encourage policies to bid positively if the current state has some importance to them; this way, if they lose the bidding, they will get some positive reward.
In the \allpay setting, by penalizing all policies, we discourage policies to bid at all unless it is absolutely important.
The \poorman setting balances these two: by neither rewarding nor penalizing the losers, we neither encourage nor discourage policies to bid.
In Section~\ref{sec:theoretical comparison of the variations}, we will see how these three settings induce different kinds of coordination through bidding.

For each policy, the bidding penalty or reward gets, respectively, subtracted or added to the \emph{nominal} reward obtained from the reward functions of the given \momdp, and the resulting reward is called the \emph{net} reward.

In summary, through this novel bidding mechanism, each policy can adjust its bid in proportion to the importance for it to execute its action in the current state, and the associated bidding penalty/reward aims to incentivize policies to be truthful.
By making the highest bidder active, it is effectively guaranteed that the most important policy is executed.
This way, we obtain a purely decentralized scheme to coordinate local policies in a given \momdp.

\begin{remark}[On the parameter $\tau$]
The parameter $\tau$ controls how frequently the agent changes its policies.
In practice, if $\tau$ is too small, the switching could be too frequent for any of the objectives to be fulfilled.
For example,  \todo{running example...}
\end{remark}


\subsection{The Design Problem and Learning Algorithms}

We consider the following learning task for our auction-based compositional framework: 
\begin{nscenter}
Given an \momdp, a constant $\tau > 0$, and $\Delta\in \{ \richman,\poorman,\allpay \}$, compute local policies that are optimal for the net rewards obtained in the mode $\Delta$, given that all other local policies behave selfishly towards maximizing their own net rewards.
\end{nscenter}

We will show how the above learning problem boils down to solving a standard learning problem in the multi-agent setting, formalized using a decentralized MDP (\decmdp) as defined below.
The only difference between a \decmdp and an \momdp (see Definition~\ref{def:MO-MDP}) is that now each reward function $R^i$ is owned by the Agent~$i$, who now controls a separate set of actions $A^i$.

\begin{definition}[\decmdp]
	 A decentralized Markov decision process (\decmdp) with \(m \in \Z_{>0}\) agents is specified by a tuple \(\mathcal{M} = (S, \mathbf{A}, T, \mathbf{R}, \mu_0)\), where
    \begin{itemize}
        \item \(S\) is the set of states,
        \item \(\mathbf{A} = \{ A^1,\ldots,A^m\} \) is a set with $A^i$ being the set of Agent~$i$'s actions,
        \item \(T : S \times A^1\times\ldots\times A^m \to \dist(S)\) is the transition function mapping a state-action pair to a distribution over the successor states,
        \item \(\mathbf{R}=\{R^i \colon S \times A^1\times\ldots\times A^m \times S \to \R_{\geq 0}\}_{i\in [1;m]}\) is the set of reward functions, and
        \item \(\mu_0 \in \dist(S)\) is the initial state distribution.
    \end{itemize}
\end{definition}
The definitions of policies and paths readily extend from \momdp to \decmdp.

Given a \decmdp, the goal is to compute an ensemble of local (memoryless) policies for all individual agents, such that for every $i\in [1;m]$, the $i$-value cannot be increased by a unanimous change of the local policy $\pi^i$.
In other words, the goal is to find a set of selfish local policies that are in a Nash equilibrium.
This is an extensively studied problem in the literature.
\todo{Do a little bit of literature survey...}

Our focus is not in improved algorithms for \decmdp, but rather to show how the local policy synthesis problem for the \momdp $\M$ in our auction-based framework reduces to the multi-agent policy synthesis problem in a \decmdp $\widetilde{\M}$.
Intuitively, for every state $s$ of $\M$, $\widetilde{\M}$ creates two kinds of copies, ones where bidding happens and are represented simply as $s$, and ones of the form $(s,t,i^*)$ that keeps track of the time $t$ elapsed since the last bidding, and the winner $i^*$ of the last bidding.
Furthermore, bidding is facilitated by extending the action space of $\M$ to include all real-valued bids, and each agent in $\widetilde{\M}$ has an identical copy of this extended action space.
After bidding in a state $s$, the winner $i^*$ is selected, and the state moves to $(s,0,i^*)$.
From this point onward, only Agent~$i^*$ selects actions $a^0,a^1,\ldots,a^\tau$ to produce the sequence $(s^1,1,i^*),(s^2,2,i^*),\ldots,(s^{\tau-1},\tau-1,i^*),s^\tau$, after which the next bidding happens, and the process repeats.
Finally, the bidding penalties or bidding rewards are only paid during the transition $s\to (s,0,i^*)$, otherwise, the rewards are inherited from the original \momdp.

We formalize this below.
Given an \momdp $\M = (S, A, T, \mathbf{R}, \mu_0)$, a constant $\tau > 0$, and the mode $\Delta\in \{ \richman,\poorman,\allpay \}$, we define the \decmdp $\widetilde{\M} = (\widetilde{S}, \widetilde{\mathbf{A}}, \widetilde{T}, \widetilde{\mathbf{R}}, \widetilde{\mu}_0)$ where
\begin{itemize}
	\item $\widetilde{S} \coloneqq S\cup S\times [0;\tau-1]\times [1;m]$,
	\item $\widetilde{\mathbf{A}}\coloneqq \{ \widetilde{A}^i\}_{i\in [1;m]}$ where $\widetilde{A}^i\coloneqq A \cup \mathbb{R}$,
	\item $\widetilde{\mu}_0\coloneqq \mu_0\times \{0\}$,
\end{itemize} 
and for every current state $s\in \widetilde{S}$ and every current action $(b^1,\ldots,b^m)\in\R^m$, writing the highest bidders as $I = \{ i\in [1;m] \mid \forall j\in [1;m]\;.\; b^i\geq b^j\}$,
\begin{itemize}
	\item $\widetilde{T}(s,b^1,\ldots,b^m) \coloneqq \text{Uniform}(\{(s,0,i)\}_{i\in I})$,
	\item $\widetilde{R}^i(s,b^1,\ldots,b^m,(s,0,i^*))\coloneqq 
	\begin{cases}
		- b^i		&	i=i^* \lor \Delta = \allpay,\\
		+ b^i		&	i\neq i^* \land \Delta = \richman,\\
		0			&	i\neq i^* \land \Delta = \poorman,
	\end{cases}	$
\end{itemize}
whereas if the current state is of the form $(s,t,i^*)\in \widetilde{S}$, for every action $(a^1,\ldots,a^m) \in A^m$,
\begin{itemize}
	\item $\widetilde{T}((s,t,i^*),a^1,\ldots,a^m) \coloneqq 
		\begin{cases}
			T(s,a^{i^*})\times \left((t+1) \mod \tau\right)	\times \{i^*\}	&	t<\tau - 1,\\
			T(s,a^{i^*})										&	t=\tau-1,
		\end{cases}
	$ 
	\item $\widetilde{R}^i((s,t,i^*),a^1,\ldots,a^m,(s',t+1,i^*))\coloneqq 
		R^i(s,a^i,s').
	$
\end{itemize}

\KM{A soundness theorem would be good, but what can we say concretely?}

\subsection{Flavors of Cooperation through Bidding}\label{sec:theoretical comparison of the variations}

We provide theoretical insights into the global behavior that emerges out of the auction-based interactions between the local policies.
For the sake of theoretical guarantees, and to be able to convey the main essence of our results, we choose the simplest bare bone setting:

\begin{assumption}
	The given \momdp has finite state and action spaces, and for every (memoryless) policy, the bottom strongly connected component (BSCC) of the resulting Markov chain (MC) is a sink state where no reward is earned.
	Furthermore, the time parameter $\tau=1$, meaning the bidding takes place at each time step before selecting the action.
\end{assumption}

Firstly, since the \momdp is finite, for each individual reward function, \emph{deterministic} memoryless policy suffices.
\todo{give some citation}

The following two types of global behaviors are of particular interest:
\begin{description}
	\item[Social welfare] is the sum (equivalently, the average) of the $i$-values for all $i$. We may ask: is the emergent global behavior guaranteed to achieve the maximal social welfare?
	\item[Fairness] is measured by the disparity between different $i$-values, i.e., $\max_{i,j\in [1;m]} |\val^i-\val^j|$. Fairness is maximized when the disparity is minimized. We may ask: is the emergent global behavior guaranteed to achieve the maximal fairness?
\end{description}

\begin{theorem}
	Suppose the \momdp is such that at each state $s$ and for every action $a$, there exists at most a single $i\in [1;m]$ such that the optimal policy for $R^i$ selects $a$ at $s$.
 	Then, the \richman setting maximizes the social welfare.
\end{theorem}

\begin{proof}[Proof sketch.]
	First, consider the simple one-shot game, where the agents bid just one time to select an action, and the reward is based on the resulting single probabilistic transition.
	Suppose for the index $i\in [1;m]$, the expected reward from using the action $a\in A$ is $E_{a}^i$, and define $E_+^i \coloneqq \max_{a\in A} E_a^i$ and $E_-^i \coloneqq \min_{a\in A} E_a^i$.
	
	We claim that the optimal bid $b^i_*$ for policy $i$ equals $(E_+^i-E_-^i)/2$, and upon winning the bidding the optimal action is $a_+=\arg\max_{a\in A} E_a^i$.
	Notice that no matter whether policy $i$ becomes the winner or the loser, its net reward is at least $(E_+^i+E_-^i)/2$: if it wins and chooses $a_+$, after paying the bidding penalty, the net reward is $E_+^i - (E_+^i-E_-^i)/2 = (E_+^i+E_-^i)/2$; if it loses, no matter what action the opponent chooses, its nominal reward is at least $E_-^i$, and after the bidding reward, the net reward is $E_-^i + (E_+^i-E_-^i)/2 = (E_+^i+E_-^i)/2$.
	If policy $i$ bids $b^i<b^i_*$, then upon losing, its net reward will be $E_-^i + b^i < E_-^i + b^i_* = (E_+^i+E_-^i)/2$.
	If it bids $b^i > b^i_*$, then upon winning, its net reward will be $E_+^i - b^i < E_+^i - b^i_* = (E_+^i+E_-^i)/2$.
	Therefore, the optimal bid is $b^i_*=(E_+^i-E_-^i)/2$, which is what each selfish policy is expected to select.
	
	Suppose, policy $i$ is the winner.
	Then, for every $j\neq i$, $b^i_*\geq b^j_*$, i.e.,  $ (E_+^i-E_-^i)/2 \geq (E_+^j-E_-^j)/2$.
	Simplifying, we get $E_+^i+E_-^j \geq E_-^i+E_+^j$.
	It follows that $E_+^i + \sum_{j\neq i} E^j \geq E_+^i + \sum_{j\neq i} E_-^j \geq E_-^i + E_+^k + \sum_{j\neq i,k} E^j$ for every for every $k\neq i$.
	Since the \momdp is purely competitive, there will be at least a single $k$ such that a given action is optimal for $k$, and therefore the claim follows for the single-shot case.
	
	Now, for the general multi-shot case, we inductively apply the above principle in the Bellman equation, which extends the claim to paths of arbitrary length. 
	The convergence of the Bellman iteration is guaranteed because it is a contraction mapping (since $\gamma < 1$).
	\KM{I am not sure about this extension.}
\end{proof}
\section{A Multi-Agent Bidding Approach for Multi-Objective RL}
\begin{definition}[\momdp]
    A multi-objective Markov decision process (\momdp) with \(m \in \Z_{>0}\) objectives is specified by a tuple \(\mathcal{M} = (S, A, T, R, \mu_0)\), where
    \begin{itemize}
        \item \(S\) is the set of states,
        \item \(A\) is the set of actions,
        \item \(T : S \times A \to \dist(S)\) is the transition function mapping a state-action pair to a distribution over states,
        \item \(R : S \times A \times S \to \R^m\) is the reward function with each output component corresponding to the different objectives, and
        \item \(\mu_0 \in \dist(S)\) is the initial state distribution.
    \end{itemize}
\end{definition}

A \textit{policy} in an \momdp~\(\mathcal{M}\) is a function \(\pi : (S \times A)^* \times S \to \dist(A)\) that maps a history of state-action pairs and the current state to a distribution over actions. 

\begin{definition}[\mabmdp]
    Let \(\mathcal{M} = (S, A, T, R, \mu_0)\) be an \momdp~with \(m\) objectives and let \(b \in \Z_{>0}\) be the bid upper bound. Also, define \(M = \{1, \dots, m\}\) be indices of the \(m\) agents corresponding to the \(m\) objectives along with \(\bot\) representing a null agent. Lastly, let \(B = \{0, \dots, b\}\) be the range of bids and \(\rho > 0\) be the bid penalty factor.
    We define the multi-agent bidding Markov decision process (\mabmdp) as a tuple \(\mathcal{B}_{\mathcal{M}} = (\hat S, \hat{A}, \hat{T}, P, \hat{R}, \hat \mu_0)\) where
    \begin{itemize}
        \item \(\hat S = M \times S\) is the new state space augmented with the index of the agent that won the previous round of bidding,
        \item \(\hat{A} = A^m \times B^m\) represents the action space of the \(m\) agents in which each agent selects an action from \(A\) and a bid from \(B\),
        \item \(\hat{T} : \hat S \times \hat{A} \to \dist(\hat S)\) is the new transition function defined as,
        \begin{equation*}
            \hat{T}((\_, s), (\vec a, \vec b)) \coloneqq \frac{1}{\abs{B_{\max}}}\sum_{i \in B_{\max}} (T(s, a_i), i)
        \end{equation*}
        where \(B_{\max} \coloneqq \{i \mid b_i = \max \{b_1, \dots, b_m\}\}\) is the set of agents with maximal bids. The tuple \((T(s, a_i), i)\) represents the distribution over \(\hat S\) induced by the original transition function \(T\) such that the second component is fixed, and the weighted sum represents taking the weighted sums of the distributions over \(\hat S\).
        \item \(P : \hat A \times M \to \R^m\) is the bidding penalty for the \(m\) agents and the second component is the index of the agent that won the bidding.
        \item \(\hat{R} : \hat S \times \hat{A} \times \hat S \to \R^m\) is the reward function for the \(m\) agents with
        \begin{equation*}
            \hat{R}_k((\_, s_0), (\vec a, \vec b), (i, s)) \coloneqq R_k(s_0, a_i, s) - P_k((\vec a, \vec b), i)
        \end{equation*}
        where \(i \in M\) is the index of the agent that won the bid and chose the action.
        \item \(\hat \mu_0 \coloneqq (\mu_0, 1)\) is the initial state distribution over \(\hat S\) induced by \(\mu_0\) and the second component is fixed to be \(1\) without loss of generality.
    \end{itemize}
\end{definition}

Given an \mabmdp~\(\mathcal{B}_{\mathcal{M}}\), a \textit{policy} for each agent indexed by \(i \in \{1, \dots, m\}\) takes a similar form: \(\pi_i : (\hat S \times \hat A)^* \times \hat S \to \hat A\). Intuitively, a state \((i, s) \in \hat S\) encodes the agent that won the bidding and chose the action to reach \(s\) in the previous step. At each step, each of the agents choose an action and a bid, and an action amongst the set of highest bidders is chosen uniformly at random. The reward function includes a penalty term that captures the desired bidding mechanism.

\section{Implementation and evaluation}

\subsection{Implementation}
Talk about:
\begin{enumerate}
    \item different bidding mechanisms
    \item choice of penalty factor
    \item action window (remarking that we could additionally allow agents to choose length of action window)
    \item use with off-the-shelf RL algorithms
\end{enumerate}

\subsection{Environments}

\subsubsection{MovingTargetsGridworld}
Important to mention that we want to maximize \(\min(\text{targets reached})\).

\subsubsection{Atari Assault}

\subsection{Baselines}
\begin{enumerate}
    \item Weighted sum of rewards with standard RL algorithms
    \item Deep W learning implemented on top of DQN
\end{enumerate}

\subsection{Performance comparison with baselines}
Include plots of training steps vs performance of our algorithms vs baselines on both environments

\subsection{Interpretability}
Include plots of distribution of control steps amongst agents, table of average, median, max, min of bids of agents

\subsection{Modularity}
Plots of performance in gridworld with increasing number of objectives

\subsection{Ablations}
Impact of max bid, penalty factor


% \subsubsection*{Broader Impact Statement}
% \label{sec:broaderImpact}
% In this optional section, RLJ/RLC encourages authors to discuss possible repercussions of their work, notably any potential negative impact that a user of this research should be aware of. 

%%%%%%%%%%%%%%%%%%%%%%%%%%%%%%%%%%%%%%%%%%%%%%%%%%%%%%%%%%%%%%%%
%% Appendices
%%%%%%%%%%%%%%%%%%%%%%%%%%%%%%%%%%%%%%%%%%%%%%%%%%%%%%%%%%%%%%%%
\appendix

% \section*{Appendix}
% % No label, since this can't be referenced meaningfully with \ref{}.
% This format should only be used if there is a single appendix (unlike in this document).

% \subsubsection*{Acknowledgments}
% \label{sec:ack}
% Use unnumbered third level headings for the acknowledgments. All acknowledgments, including those to funding agencies, go at the end of the paper. Only add this information once your submission is accepted and deanonymized. The acknowledgments do not count towards the 8--12 page limit.

%%%%%%%%%%%%%%%%%%%%%%%%%%%%%%%%%%%%%%%%%%%%%%%%%%%%%%%%%%%%%%%%
%% NOTE: THIS MARKS THE END OF THE "MAIN TEXT"
%%%%%%%%%%%%%%%%%%%%%%%%%%%%%%%%%%%%%%%%%%%%%%%%%%%%%%%%%%%%%%%%

%%%%%%%%%%%%%%%%%%%%%%%%%%%%%%%%%%%%%%%%%%%%%%%%%%%%%%%%%%%%%%%%
%% Bibliography
%%%%%%%%%%%%%%%%%%%%%%%%%%%%%%%%%%%%%%%%%%%%%%%%%%%%%%%%%%%%%%%%
\bibliography{main}
\bibliographystyle{rlj}

%%%%%%%%%%%%%%%%%%%%%%%%%%%%%%%%%%%%%%%%%%%%%%%%%%%%%%%%%%%%%%%%
% AUTHOR: If your paper has no supplementary materials, you may 
%         comment out the line below, which creates the title for
%         the supplementary materials.
%%%%%%%%%%%%%%%%%%%%%%%%%%%%%%%%%%%%%%%%%%%%%%%%%%%%%%%%%%%%%%%%
% \beginSupplementaryMaterials

% Content that appears after the references are not part of the ``main text,'' have no page limits, are not necessarily reviewed, and should not contain any claims or material central to the paper. 
% %
% If your paper includes supplementary materials, use the \begin{center}
%     {\tt {\textbackslash}beginSupplementaryMaterials} 
% \end{center}
% command as in this example, which produces the title and disclaimer above. 
% %
% If your paper does not include supplementary materials, this command can be removed or commented out.

\end{document}
